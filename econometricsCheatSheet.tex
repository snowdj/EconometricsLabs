\documentclass[10pt,landscape]{article}
\usepackage{multicol}
\usepackage{calc}
\usepackage{ifthen}
\usepackage{amsmath}
\usepackage{amsfonts}
\usepackage{lipsum}
\usepackage{booktabs}
\usepackage{xcolor}
\usepackage{threeparttable}
\usepackage{stackengine}
\usepackage{scalerel}
\usepackage[landscape]{geometry}

\definecolor{oucrimson}{RGB}{132,22,23} % OU crimson

\usepackage{hyperref}
\hypersetup{
    unicode=true,
    bookmarksnumbered=true,
    bookmarksopen=true,
    bookmarksopenlevel=3,
    breaklinks=true, 
    pdfborder={0 0 0},
    colorlinks,
    citecolor=oucrimson,
    filecolor=oucrimson,
    linkcolor=oucrimson,
    urlcolor=oucrimson
}


% To make this come out properly in landscape mode, do one of the following
% 1.
%  pdflatex latexsheet.tex
%
% 2.
%  latex latexsheet.tex
%  dvips -P pdf  -t landscape latexsheet.dvi
%  ps2pdf latexsheet.ps


% If you're reading this, be prepared for confusion.  Making this was
% a learning experience for me, and it shows.  Much of the placement
% was hacked in; if you make it better, let me know...


% This sets page margins to .5 inch if using letter paper, and to 1cm
% if using A4 paper. (This probably isn't strictly necessary.)
% If using another size paper, use default 1cm margins.
\ifthenelse{\lengthtest { \paperwidth = 11in}}
	{ \geometry{top=.5in,left=.5in,right=.5in,bottom=.5in} }
	{\ifthenelse{ \lengthtest{ \paperwidth = 297mm}}
		{\geometry{top=1cm,left=1cm,right=1cm,bottom=1cm} }
		{\geometry{top=1cm,left=1cm,right=1cm,bottom=1cm} }
	}

% Turn off header and footer
\pagestyle{empty}
 

% Redefine section commands to use less space
\makeatletter
\renewcommand{\section}{\@startsection{section}{1}{0mm}%
                                {-1ex plus -.5ex minus -.2ex}%
                                {0.5ex plus .2ex}%x
                                {\normalfont\large\bfseries}}
\renewcommand{\subsection}{\@startsection{subsection}{2}{0mm}%
                                {-1explus -.5ex minus -.2ex}%
                                {0.5ex plus .2ex}%
                                {\normalfont\normalsize\bfseries}}
\renewcommand{\subsubsection}{\@startsection{subsubsection}{3}{0mm}%
                                {-1ex plus -.5ex minus -.2ex}%
                                {1ex plus .2ex}%
                                {\normalfont\small\bfseries}}
\makeatother

% Define BibTeX command
\def\BibTeX{{\rm B\kern-.05em{\sc i\kern-.025em b}\kern-.08em
    T\kern-.1667em\lower.7ex\hbox{E}\kern-.125emX}}

% Don't print section numbers
\setcounter{secnumdepth}{0}


\setlength{\parindent}{0pt}
\setlength{\parskip}{0pt plus 0.5ex}

% Really wide hats
\stackMath
\newcommand\reallywidehat[1]{%
\savestack{\tmpbox}{\stretchto{%
  \scaleto{%
    \scalerel*[\widthof{\ensuremath{#1}}]{\kern-.6pt\bigwedge\kern-.6pt}%
    {\rule[-\textheight/2]{1ex}{\textheight}}%WIDTH-LIMITED BIG WEDGE
  }{\textheight}% 
}{0.5ex}}%
\stackon[1pt]{#1}{\tmpbox}%
}


% -----------------------------------------------------------------------

\begin{document}

\raggedright
\footnotesize
\begin{multicols}{3}


% multicol parameters
% These lengths are set only within the two main columns
%\setlength{\columnseprule}{0.25pt}
\setlength{\premulticols}{1pt}
\setlength{\postmulticols}{1pt}
\setlength{\multicolsep}{1pt}
\setlength{\columnsep}{2pt}

\begin{center}
     {\Large{\textbf{Econometrics Cheat Sheet}}} \\
     
     {\footnotesize by Tyler Ransom, University of Oklahoma}
     
     {\footnotesize \href{https://www.twitter.com/tyleransom}{\makeatletter @tyleransom \makeatother}}
\end{center}

\section{Data \& Causality}
Basics about data types and causality.

\subsection{Types of data}
\newlength{\MyLen}
\settowidth{\MyLen}{\texttt{letterpaper}/\texttt{a4paper} \ }
\begin{tabular}{@{}p{\the\MyLen}%
                @{}p{\linewidth-\the\MyLen}@{}}
Experimental            & Data from randomized experiment \\
Observational           & Data collected passively \\
Cross-sectional         & Multiple units, one point in time \\
Time series             & Single unit, multiple points in time \\
Longitudinal (or Panel) & Multiple units followed over multiple time periods \\
\end{tabular}

\textbf{Experimental data} 
\begin{itemize}
\item Correlation $\implies$ Causality
\item Very rare in Social Sciences
\end{itemize}






\section{Statistics basics}
We examine a \textbf{random sample} of data to learn about the population

\medskip{}

\begin{tabular}{@{}p{\the\MyLen}%
                @{}p{\linewidth-\the\MyLen}@{}}
Random sample         &  Representative of population \\
Parameter $(\theta)$  &  Some number describing population  \\
Estimator of $\theta$ &  Rule assigning value of $\theta$ to sample    \\
                      &  e.g. Sample average, $\overline{Y} = \frac{1}{N}\sum_{i=1}^{N} Y_{i}$ \\
Estimate of $\theta$  &  What the estimator spits out \par
                         for a particular sample $(\hat{\theta})$   \\
Sampling distribution & Distribution of estimates \par
                        across all possible samples \\
Bias of estimator $W$ & $E\left(W\right)-\theta$ \\
Efficiency            & $W$ efficient if $Var(W)<Var(\widetilde{W})$ \\
Consistency           & $W$ consistent if $\hat{\theta}\rightarrow\theta$ as $N\rightarrow\infty$ \\
\end{tabular}

\subsection{Hypothesis testing}
The way we answer yes/no questions about our population using a sample of data. e.g. ``Does increasing public school spending increase student achievement?''

\medskip{}

\begin{tabular}{@{}p{\the\MyLen}%
                @{}p{\linewidth-\the\MyLen}@{}}
null hypothesis $(H_0)$       & Typically, $H_0: \theta=0$ \\
alt. hypothesis $(H_a)$       & Typically, $H_0: \theta\neq 0$ \\
significance level $(\alpha)$ & Tolerance for making Type I error; \par (e.g. 10\%, 5\%, or 1\%)\\
test statistic $(T)$          & Some function of the sample of data \\
critical value $(c)$          & Value of $T$ such that reject $H_0$ if $\vert T \vert > c$; \par $c$ depends on $\alpha$; \par $c$ depends on if 1- or 2-sided test\\
$p$-value                     & Largest $\alpha$ at which fail to reject $H_0$; \par reject $H_0$ if $p<\alpha$ \\
\end{tabular}

% \medskip{}

% \begin{align*}
% t = & \frac{\text{estimate} - \text{null}}{\text{std. err.}} \\
% \end{align*}

\section{Simple Regression Model}
Regression is useful because we can estimate a \textit{ceteris paribus} relationship between some variable $x$ and our outcome $y$

\begin{align*}
y=\beta_{0}+\beta_{1}x+u
\end{align*}

We want to estimate $\hat{\beta}_1$, which gives us the effect of $x$ on $y$.

\subsection{OLS formulas}
To estimate $\hat{\beta}_0$ and $\hat{\beta}_1$, we make two assumptions:
\begin{enumerate}
\item $E\left(u\right) = 0$
\item $E\left(u\vert x\right) = E\left(u\right)$ for all $x$
\end{enumerate}

When these hold, we get the following formulas:
\begin{align*}
\hat{\beta}_{0}&=\overline{y}-\hat{\beta}_{1}\overline{x} \\
\hat{\beta}_{1}&=\frac{\reallywidehat{Cov}\left(y,x\right)}{\reallywidehat{Var}\left(x\right)}
\end{align*}

\begin{tabular}{@{}p{\the\MyLen}%
                @{}p{\linewidth-\the\MyLen}@{}}
fitted values $(\hat{y}_i)$     & $\hat{y}_{i} = \hat{\beta}_0 + \hat{\beta}_1 x_{i}$ \\
residuals $(\hat{u}_i)$         & $\hat{u}_{i} = y_{i} - \hat{y}_i$ \\
Total Sum of Squares            & $SST=\sum_{i=1}^{N}(y_{i}-\overline{y})^{2}$ \\
Expl. Sum of Squares            & $SSE =\sum_{i=1}^{N}(\hat{y}_{i}-\overline{y})^{2}$\\
Resid. Sum of Squares           & $SSR =\sum_{i=1}^{N}\hat{u}_{i}^{2}$\\
$R$-squared ($R^2$)             & $R^{2}=\frac{SSE}{SST}$; \par ``frac. of var. in $y$ explained by $x$''\\
\end{tabular}

\subsection{Algebraic properties of OLS estimates}
\begin{enumerate}
    \item[] $\sum_{i=1}^{N}\hat{u}_{i}=0$ (mean \& sum of residuals is zero)
    \item[] $\sum_{i=1}^{N}x_{i}\hat{u}_{i}=0$ (zero covariance bet. $x$ and resids.)
    \item[] The OLS line (SRF) always passes through $\left(\overline{x},\overline{y}\right)$
    \item[] $SSE + SSR = SST$
    \item[] $0 \leq R^2 \leq 1$
\end{enumerate}

\subsection{Interpretation and functional form}
\begin{itemize}
    \item[] Our model is restricted to be \textbf{linear in parameters}
    \item[] But not linear in $x$
    \item[] Other functional forms can give more realistic model
\end{itemize}

\begin{center}
\begin{threeparttable}
\begin{tabular}{@{}l@{}c@{}c@{}l@{}}
    \toprule
    Model       & DV        & RHS        & Interpretation of $\beta_1$ \\ \midrule
    Level-level & $y$       & $x$         & $\phantom{\%}\Delta y = \beta_1 \Delta x$ \\
    Level-log   & $y$       & $\phantom{\%}\log(x)\phantom{\%}$   & $\phantom{\%}\Delta y \approx \left(\beta_1 / 100\right)\left[1\%\Delta x\right]\phantom{\%}$ \\
    Log-level   & $\phantom{\%}\log(y)\phantom{\%}$ & $x$         & $\%\Delta y \approx \left(100\beta_1\right)\Delta x$ \\
    Log-log     & $\log(y)$ & $\log(x)$   & $\%\Delta y \approx \beta_1 \%\Delta x$ \\ 
    Quadratic   & $y$       & $x+x^2$     & $\phantom{\%}\Delta y = \left(\beta_1+2\beta_2 x\right)\Delta x$ \\ \bottomrule
\end{tabular}
\scriptsize Note: $\text{DV }=\text{ dependent variable}$; $\text{RHS }=\text{ right hand side}$
\end{threeparttable}
\end{center}

\section{Multiple Regression Model}
Multiple regression is more useful than simple regression because we can more plausibly estimate \textit{ceteris paribus} relationships (i.e. $E\left(u\vert x\right) = E\left(u\right)$ is more plausible)
\begin{align*}
y=\beta_{0}+\beta_{1}x_1+\cdots+\beta_{k}x_k+u
\end{align*}

$\hat{\beta}_1,\ldots,\hat{\beta}_k$: \textbf{partial effect} of each of the $x$'s on $y$

\begin{align*}
\hat{\beta}_{0}&=\overline{y}-\hat{\beta}_{1}\overline{x}_1-\cdots-\hat{\beta}_{k}\overline{x}_k \\
\hat{\beta}_{j}&=\frac{\reallywidehat{Cov}\left(y,\text{residualized }x_{j}\right)}{\reallywidehat{Var}\left(\text{residualized }x_{j}\right)}
\end{align*}

where ``residualized $x_j$'' means the residuals from OLS regression of $x_j$ on all other $x$'s (i.e. $x_1, \ldots, x_{j-1}, x_{j+1}, \ldots x_k$)
    
\subsection{Gauss-Markov Assumptions}
\begin{enumerate}
    \item $y$ is a linear function of the $\beta$'s
    \item $y$ and $x$'s are randomly sampled from population
    \item No perfect multicollinearity
    \item $E\left(u\vert x_1, \ldots, x_k\right) = E\left(u\right) = 0$ (Unconfoundedness)
    \item $Var\left(u\vert x_1, \ldots, x_k\right) = Var\left(u\right) = \sigma^2$ (Homoskedasticity)
\end{enumerate}

\begin{tabular}{@{}p{\the\MyLen}%
                @{}p{\linewidth-\the\MyLen}@{}}
When (1)-(4) hold: & OLS is unbiased; i.e. $E(\hat{\beta}_j) = \beta_j$ \\
When (1)-(5) hold: & OLS is Best Linear Unbiased Estimator \\
\end{tabular}

\subsection{Variance of $u$ (a.k.a. ``error variance'')}
\begin{align*}
\hat{\sigma}^2 &= \frac{SSR}{N-K-1} \\
               &= \frac{1}{N-K-1}\sum_{i=1}^N \hat{u}_i^2
\end{align*}

\subsection{Variance and Standard Error of $\hat{\beta}_j$}
\begin{equation*}
Var(\hat{\beta}_{j})=\frac{\sigma ^{2}}{SST_{j}(1-R_{j}^{2})}\text{, }%
j=1,2,...,k
\end{equation*}
where
\begin{align*}
SST_{j}&=(N-1)Var(x_j) = \sum_{i=1}^N (x_{ij} - \overline{x}_j)\\
R_{j}^{2} &= R^2 \text{ from a regression of } x_j \text{ on all other } x \text{'s}
\end{align*}
\begin{tabular}{@{}p{\the\MyLen}%
                @{}p{\linewidth-\the\MyLen}@{}}
Standard deviation: & $\sqrt{Var}$ \\
Standard error:     & $\sqrt{\reallywidehat{Var}}$ \\
\end{tabular}

\begin{equation*}
se(\hat{\beta}_{j})=\sqrt{\frac{\hat{\sigma} ^{2}}{SST_{j}(1-R_{j}^{2})}}, j=1,\ldots,k
\end{equation*}

\subsection{Classical Linear Model (CLM)}
Add a 6th assumption to Gauss-Markov:
\begin{enumerate}
    \setcounter{enumi}{5}
    \item $u$ is distributed $N\left(0,\sigma^2\right)$
\end{enumerate}

Need this to know what the exact \textit{distribution} of $\hat{\beta}_j$ is

\begin{itemize}
    \item If A(6) fails, need \textbf{asymptotics} to test  $\beta$'s
    \item Then, interpret distr. of $\hat{\beta}_j$ as asymptotic (not exact)
\end{itemize}



\section{Testing Hypotheses about the $\beta$'s}
\begin{itemize}
    \item Under A (1)--(6), can test hypotheses about the $\beta$'s
    \item Or, (much more plausible) A (1)--(5) $+$ asymptotics
\end{itemize}

\subsection{$t$-test for simple hypotheses}
To test a simple hypothesis like
\begin{align*}
H_0&:\beta_j = 0\\
H_a&: \beta_j \neq 0
\end{align*}
use a $t$-test:
\begin{align*}
t&=\frac{\hat{\beta}_j-0}{se\left(\hat{\beta}_j\right)}
\end{align*}
where $0$ is the null hypothesized value.

\medskip{}

Reject $H_0$ if $p<\alpha$ or if $\vert t \vert > c$ (See: Hypothesis testing)

\subsection{$F$-test for joint hypotheses}
Can't use a $t$-test for joint hypotheses, e.g.:
\begin{align*}
H_{0}&:\beta _{3}=0\text{, }\beta _{4}=0\text{, }\beta _{5}=0\\
H_{a}&:\beta _{3}\neq0\text{ OR }\beta _{4}\neq0\text{ OR }\beta _{5}\neq0
\end{align*}
Instead, use $F$ statistic:
\begin{align*}
F=\frac{(SSR_{r}-SSR_{ur})/(df_{r}-df_{ur})}{SSR_{ur}/df_{ur}}=\frac{(SSR_{r}-SSR_{ur})/q}{SSR_{ur}/(N-k-1)}
\end{align*}
where
\begin{align*}
SSR_r    &= SSR \text{ of restricted model (if }H_0\text{ true)} \\
SSR_{ur} &= SSR \text{ of unrestricted model (if }H_0\text{ false)} \\
q        &= \text{\# of equalities in }H_0 \\
N-k-1    &= \text{Deg. Freedom of unrestricted model} \\
\end{align*}
Reject $H_0$ if $p<\alpha$ or if $F > c$ (See: Hypothesis testing)

\medskip{}

Note: $F>0$, always

\section{Qualitative data}
\begin{itemize}
    \item Can use qualitative data in our model
    \item Must create a \textbf{dummy variable}
    \item e.g. ``Yes'' represented by 1 and ``No'' by 0
\end{itemize}

\textbf{dummy variable trap:} Perfect collinearity that happens when too many dummy variables are included in the model
\begin{align*}
y &= \beta_0 + \beta_1 happy + \beta_2 not\_happy + u
\end{align*}

The above equation suffers from the dummy variable trap because units can only be ``happy'' or ``not happy,'' so including both would result in perfect collinearity with the intercept

\subsection{Interpretation of dummy variables}
Interpretation of dummy variable coefficients is always relative to the excluded category (e.g. $not\_happy$):
\begin{align*}
y &= \beta_0 + \beta_1 happy + \beta_2 age + u
\end{align*}

\begin{itemize}
    \item[$\beta_1$:] avg. $y$ for those who are happy \textit{compared to} those who are unhappy, holding fixed age 
\end{itemize}

\subsection{Interaction terms}
\textbf{interaction term:} When two $x$'s are multiplied together
\begin{align*}
y &= \beta_0 + \beta_1 happy + \beta_2 age + \beta_3 happy \times age + u
\end{align*}

\begin{itemize}
    \item[$\beta_3$:] difference in $age$ \textbf{slope} for those who are happy \textit{compared to} those who are unhappy
\end{itemize}

\subsection{Linear Probability Model (LPM)}
When $y$ is a dummy variable, e.g.
\begin{align*}
happy &= \beta_0 + \beta_1 age + \beta_2 income + u
\end{align*}

$\beta$'s are interpreted as \textit{change in probability}:
\begin{align*}
\Delta\Pr(y=1) = \beta_1\Delta x
\end{align*}
By definition, homoskedasticity is violated in the LPM




\section{Time Series (TS) data}
\begin{itemize}
    \item Observe one unit over many time periods
    \item e.g. US quarterly GDP, 3-month T-bill rate, etc.
    \item New G-M assumption: no serial correlation in $u_t$
    \item Remove random sampling assumption (makes no sense)
\end{itemize}

\subsection{Two focuses of TS data}
\begin{enumerate}
    \item Causality (e.g. $\uparrow$ taxes $\overset{?}{\implies}$ $\downarrow$ GDP growth)
    \item Forecasting (e.g. AAPL stock price next quarter?)
\end{enumerate}

\subsection{Requirements for TS data}
To properly use TS data for causal inf / forecasting,\\need data free of the following elements:

\medskip{}

\begin{tabular}{@{}p{\the\MyLen}%
                @{}p{\linewidth-\the\MyLen}@{}}
Trends: & $y$ always $\uparrow$ or $\downarrow$ every period \\
Seasonality: & $y$ always $\uparrow$ or $\downarrow$ at regular intervals \\
Non-stationarity:  & $y$ has a unit root; i.e. not stable \\
\end{tabular}

Otherwise, $R^2$ and $\hat{\beta}_{j}$'s are misleading

\subsection{$AR(1)$ and Unit Root Processes}
$AR(1)$ model (Auto Regressive of order 1): 
\begin{align*}
y_t = \rho y_{t-1} + u_{t}
\end{align*}
Stable if $\vert\rho\vert<1$; Unit Root if $\vert\rho\vert\geq 1$

``Non-stationary,'' ``Unit Root,'' ``Integrated'' are all synonymous
\subsubsection{Correcting for Non-stationarity}
Easiest way is to take a first difference:

\medskip{}

\begin{tabular}{@{}p{\the\MyLen}%
                @{}p{\linewidth-\the\MyLen}@{}}
First difference: & Use  $\Delta y = y_{t}-y_{t-1}$ instead of $y_t$ \\
Test for unit root: & Augmented Dickey-Fuller (ADF) test \\
$H_0$ of ADF test: & $y$ has a unit root \\
\end{tabular}


\subsection{TS Forecasting}
A good forecast minimizes \textbf{forecasting error} $\hat{f}_t$:
\begin{align*}
\min_{f_t} E\left(e_{t+1}^{2}\vert I_t\right) = E\left[\left(y_{t+1}-f_t\right)^{2}\vert I_t\right]
\end{align*}
where $I_t$ is the \textbf{information set}

% \medskip{}

% Want a model that predicts the future as correctly as possible

\medskip{}

RMSE measures forecast performance (on future data): 
\begin{align*}
\text{Root Mean Squared Error} = \sqrt{\frac{1}{m}\sum_{h=0}^{m-1} \hat{e}^2_{T+h+1} }
\end{align*}

\medskip{}

Model with lowest RMSE is best forecast

\begin{itemize}
    \item Can choose $f_{t}$ in many ways
    \item Basic way: $\hat{y}_{T+1}$ from linear model
    \item ARIMA, ARMA-GARCH are cutting-edge models
\end{itemize}

\subsubsection{Granger causality}
$z$ \textbf{Granger causes} $y$ if, after controlling for past values of $y$, past values of $z$ help forecast $y_t$


\section{CLM violations}
\subsubsection{Heteroskedasticity}
\begin{itemize}
    \item Test: Breusch-Pagan or White tests ($H_0:$ homosk.)
    \item If $H_0$ rejected, SEs, t-, and F-stats are invalid
    \item Instead use heterosk.-robust SEs and t- and F-stats
\end{itemize}

\subsubsection{Serial correlation}
\begin{itemize}
    \item Test: Breusch-Godfrey test ($H_0:$ no serial corr.)
    \item If $H_0$ rejected, SEs, t-, and F-stats are invalid
    \item Instead use HAC SEs and t- and F-stats
    \item HAC: ``Heterosk. and Autocorrelation Consistent''
\end{itemize}

\subsubsection{Measurement error}
\begin{itemize}
    \item Measurement error in $x$ can be a violation of A4
    \item \textbf{Attenuation bias:} $\beta_j$ biased towards 0
\end{itemize}

\subsubsection{Omitted Variable Bias}
When an important $x$ is excluded: \textbf{omitted variable bias}.

\medskip{}

Bias depends on two forces:
\begin{enumerate}
    \item Partial effect of $x_2$ on $y$ (i.e. $\beta_2$)
    \item Correlation between $x_2$ and $x_1$
\end{enumerate}

Which direction does the bias go?
\begin{center}
\begin{threeparttable}
\begin{tabular}{@{}l@{}l@{}l@{}}
\toprule
& $Corr(x_{1},x_{2})>0$\phantom{\%\%\%} & $Corr(x_{1},x_{2})<0$\phantom{\%\%\%} \\ \midrule
$\beta _{2}>0$\phantom{\%\%\%} & Positive Bias & Negative Bias \\
$\beta _{2}<0$ & Negative Bias & Positive Bias \\ \bottomrule
\end{tabular}
\footnotesize Note: ``Positive bias'' means $\beta_1$ is too big; \par
``Negative bias'' means $\beta_1$ is too small
\end{threeparttable}
\end{center}

\subsection{How to resolve $E\left(u\vert\mathbf{x}\right)\neq 0$}
How can we get unbiased $\hat{\beta}_j$'s when $E\left(u\vert\mathbf{x}\right)\neq 0$?

\medskip{}

\begin{itemize}
    \item Include lagged $y$ as a regressor
    \item Include proxy variables for omitted ones
    \item Use instrumental variables
    \item Use a natural experiment (e.g. diff-in-diff)
    \item Use panel data
\end{itemize}



\subsubsection{Instrumental Variables (IV)}
A variable $z$, called the instrument, satisfies:

\medskip{}

\begin{enumerate}
    \item $cov\left(z,u\right) = 0$ (\textbf{not} testable)
    \item $cov\left(z,x\right) \neq 0$ (testable)
\end{enumerate}

\medskip{}

$z$ typically comes from a \textbf{natural experiment}

\begin{align*}
    \hat{\beta}_{IV} = \frac{cov\left(z,y\right)}{cov\left(z,x\right)}
\end{align*}

\begin{itemize}
    \item SE's much larger when using IV compared to OLS
    \item Be aware of \textbf{weak instruments}
\end{itemize}

\textbf{When there are multiple instruments:}

\begin{itemize}
    \item use Two-stage least squares (2SLS)
    \item exclude at least as many $z$'s as endogenous $x$'s
    \item[1st stage:] regress endogenous $x$ on $z$'s and exogenous $x$'s
    \item[2nd stage:] regress $y$ on $\hat{x}$ and exogenous $x$'s
\end{itemize}

\textbf{Test for weak instruments:} Instrument is weak if

\begin{itemize}
    \item 1st stage $F$ stat $< 10$
    \item or 1st stage $\vert t \vert < \sqrt{10}\approx 3.2$
\end{itemize}

\subsubsection{Difference in Differences (DiD)}
Can get causal effects from \textbf{pooled cross sectional data}

\medskip{}

A nat. experiment divides units into treatment, control groups

\begin{align*}
    y_{it} = \beta_0 + \delta_0 d2_t + \beta_1 dT_{it} + \delta_1 d2_{t}\times dT_{it} + u_{it}
\end{align*}
where
\begin{itemize}
    \item $d2_t=$ dummy for being in time period 2
    \item $dT_{it}=$ dummy for being in the treatment group
    \item $\hat{\delta}_1=$ \textbf{difference in differences}
\end{itemize}
\begin{align*}
    \hat{\delta}_1 = \left(\overline{y}_{treat,2}-\overline{y}_{control,2}\right) - \left(\overline{y}_{treat,1}-\overline{y}_{control,1}\right)
\end{align*}

\medskip{}


\textbf{Extensions:}
\begin{itemize}
    \item Can also include $x$'s in the model
    \item Can also use with more than 2 time periods
    \item $\hat{\delta}_1$ has same interpretation, different math formula
\end{itemize}

\medskip{}

\textbf{Validity:}
\begin{itemize}
    \item Need $y$ changing across time and treatment for reasons only due to the policy
    \item a.k.a. \textbf{parallel trends assumption}
\end{itemize}


\subsubsection{Panel data}
Follow same sample of units over multiple time periods
\begin{align*}
    y_{it} = \beta_0 + \beta_1 x_{it1} + \cdots + \beta_k x_{itk} + \underbrace{a_i + u_{it}}_{\nu_{it}}
\end{align*}
\begin{itemize}
    \item $\nu_{it}=$ \textbf{composite error}
    \item $a_i=$  unit-specific unobservables
    \item $u_{it}=$ idiosyncratic error
    \item Allow $E\left(a\vert\mathbf{x}\right)\neq 0$
    \item Maintain $E\left(u\vert\mathbf{x}\right)= 0$
\end{itemize}

\medskip{}

Four different methods of estimating $\beta_j$'s:
\begin{enumerate}
    \item Pooled OLS (i.e. ignore composite error)
    \item First differences (FD):
\begin{align*}
    \Delta y_{i} = \beta_1 \Delta x_{i1} + \cdots + \Delta \beta_k x_{ik} + \Delta u_{i}
\end{align*}
estimated via Pooled OLS on transformed data
    \item Fixed effects (FE):
\begin{align*}
    y_{it}-\overline{y}_{i} &= \beta_1 \left(x_{it1}-\overline{x}_{i1}\right) + \cdots \\
    &+\beta_k \left( x_{itk}-\overline{x}_{ik}\right) + \left(u_{it}-\overline{u}_{i}\right)
\end{align*}
estimated via Pooled OLS on transformed data
    \item Random effects (RE):
\begin{align*}
    y_{it}-\theta\overline{y}_{i} &= \beta_0 \left(1-\theta\right) + \beta_1 \left(x_{it1}-\theta\overline{x}_{i1}\right) + \cdots \\
    &+\beta_k \left( x_{itk}-\theta\overline{x}_{ik}\right) + \left(\nu_{it}-\theta\overline{\nu}_{i}\right)
\end{align*}
estimated via FGLS, where
\begin{align*}
    \theta &= 1-\sqrt{\frac{\sigma^2_u}{\sigma^2_u + T\sigma^2_a}} \\
    \hat{\beta}_{RE} \rightarrow \hat{\beta}_{FE} & \text{ as } \theta \rightarrow 1 \\
    \hat{\beta}_{RE} \rightarrow \hat{\beta}_{POLS} & \text{ as } \theta \rightarrow 0 \\
\end{align*}
\textbf{RE assumes $E\left(a\vert\mathbf{x}\right)= 0$}
\end{enumerate}

\paragraph{Cluster-robust SEs}
\begin{itemize}
    \item Serial correlation of $\nu_{it}$ is a problem
    \item Use \textbf{cluster-robust} SEs
    \item These correct for serial corr. and heterosk.
    \item Cluster at the unit level
\end{itemize}


\section{Binary dependent variables}
Three options for estimation when $y$ is binary (0/1):
\begin{itemize}
    \item Linear Probability Model
    \item Logit
    \item Probit
\end{itemize}

Latter two are \textit{nonlinear} models:
\begin{align*}
    P\left(y=1\,\vert\,\mathbf{x}\right)&=G\left(\beta _{0}+\beta_{1}x_{1}+\beta_{2}x_{2}\right)
\end{align*}
where $G\left(\cdot\right)$ is some nonlinear function satisfying $0<G\left(\cdot\right)<1$

\subsection{Trade-offs with logit/probit}
\textbf{Disadvantages}
\begin{itemize}
\item Now it's much harder to estimate and interpret $\beta$'s!
\item Can no longer use OLS; instead use maximum likelihood
\item Nonlinear $G\left(\cdot\right)\implies$ 
\begin{itemize}
\item Must use chain rule to compute slope
\item Slope of tangent line depends on $\mathbf{x}$!
\end{itemize}
\end{itemize}
\textbf{Main advantage}
\begin{itemize}
\item Now $0 < \widehat{y} < 1 \implies$ more realistic
\item (Recall: in LPM, possible to have negative probabilities)
\end{itemize}

\subsection{Common choices for $G\left(\cdot\right)$}
\textbf{Logit model:}
\begin{align*}
G\left(\mathbf{x}\beta\right)&=\frac{\exp\left(\mathbf{x}\beta\right)}{1+\exp\left(\mathbf{x}\beta\right)}=\Lambda\left(\mathbf{x}\beta\right)
\end{align*}


\textbf{Probit model:}
\begin{align*}
G\left(\mathbf{x}\beta\right)&= \int_{-\infty}^{\mathbf{x}\beta}\phi\left(z\right)dz =\Phi\left(\mathbf{x}\beta\right)
\end{align*}
where $\phi\left(\cdot\right)$ is the standard normal pdf

\subsection{Interpreting logit/probit parameter estimates}
\begin{itemize}
\item $\beta$'s that come from logit/probit $\neq$ $\beta$'s from LPM
\item But, \textit{sign} is same
\item In logit/probit, we have
\begin{align*}
\frac{\partial p\left(\mathbf{x}\right)}{\partial x_{j}}=\beta _{j}g\left(\mathbf{x}\beta\right)
\end{align*}
where $g\left(\mathbf{x}\beta\right)$ is the first derivative of $G\left(\mathbf{x}\beta\right)$
\item In LPM, we have $\frac{\partial p\left(\mathbf{x}\right)}{\partial x_{j}}=\beta _{j}$
% \begin{align*}
% \frac{\partial p\left(\mathbf{x}\right)}{\partial x_{j}}=\beta _{j}
% \end{align*}
\end{itemize}


% \textcolor{white}{\lipsum[1-3]}

\rule{0.3\linewidth}{0.25pt}
\scriptsize

Layout by Winston Chang, \href{http://wch.github.io/latexsheet/}{http://wch.github.io/latexsheet/}
\end{multicols}
\end{document}
